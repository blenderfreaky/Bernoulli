\documentclass[10pt,a4paper, twoside]{article}
\usepackage{amsmath}
\usepackage{amssymb}
\usepackage[ngerman]{babel}
\usepackage{graphicx}
\usepackage{esvect}
\usepackage{amsfonts}
\usepackage[dvips]{epsfig}
\usepackage{xcolor}
\usepackage{amsthm}
\usepackage{tikz}
\usepackage{titlesec}
\usepackage{amsthm}
\usepackage{comment} 

\usepackage{mathtools} 
\usepackage{pgf}
%\usepackage{mathrsfs}
\usetikzlibrary{arrows}
\usepackage{enumitem}
\usepackage{scrextend}    

\usepackage{caption}
\usepackage{tkz-euclide}
\usetikzlibrary{intersections, through}
\usepackage{listings}
\newcommand{\norm}[1]{\lvert #1 \rvert}
\newcommand{\eqtext}[1]{\ensuremath{\stackrel{\text{#1}}{=}}}
\newcommand{\colvec}[2]{\begin{pmatrix}#1\\#2\end{pmatrix}}
\def\cul#1#2{\color{#1}\underline{{\color{black}#2}}\color{black}}
\newcounter{oursec}
\renewcommand*\theoursec{\arabic{oursec}. }
\newcounter{oursubsec}[oursec]
\renewcommand*\theoursubsec{\arabic{oursec}.\arabic{oursubsec}. }
\newcommand{\oursec}[1]{\vspace{4ex}\stepcounter{oursec}%
{\Large\bf\theoursec #1\\[2ex]}}
\newcommand{\oursubsec}[1]{\vspace{2ex}\stepcounter{oursubsec}%
{\large\bf\theoursubsec #1\\[1ex]}}





\titleformat*{\section}{\Large\bfseries}
\titleformat*{\subsection}{\Large\bfseries}
\setlength{\parindent}{0mm}
\setlength{\parskip}{1.2ex}




\oddsidemargin0,95cm
\evensidemargin0,95cm
\textwidth16cm
\textheight22.5cm
\topmargin-1.5cm
\hoffset-12mm
\parskip2ex
\parindent0cm

\theoremstyle{definition}
\newtheorem*{Beispiele}{Beispiele}

\newtheorem{Beispiel}{Beispiel}[subsection]
\renewcommand{\theBeispiel}{\arabic{Beispiel}}

\newtheorem*{Aufgabe}{Aufgabe}
\newtheorem*{Ueberlegung}{\"Uberlegung}
\newtheorem*{Bezeichnung}{Bezeichnung}
\newtheorem*{Algorithmus}{Algorithmus}
\newtheorem*{Ansatz}{Ansatz}
\newtheorem*{Behauptung}{Behauptung}

\newtheorem{Korollar}{Korollar}[subsection]
\renewcommand{\theKorollar}{\arabic{Korollar}}
\newtheorem{Definition}{Definition}[subsection]
\renewcommand{\theDefinition}{\arabic{Definition}}
\newtheorem{Bemerkung}{Bemerkung}[subsection]
\renewcommand{\theBemerkung}{\arabic{Bemerkung}}
\newenvironment{Beweis}{\noindent\textbf{Beweis:}}{\hfill $\Box$ }


\theoremstyle{plain}

\newtheorem{Satz}{Satz}[subsection]
\renewcommand{\theSatz}{\arabic{Satz}}
\newtheorem{Lemma}{Lemma}[subsection]
\renewcommand{\theLemma}{\arabic{Lemma}}



\numberwithin{equation}{subsection}
\renewcommand{\theequation}{\arabic{equation}}


%% zusï¿œtzliche Befehle
\newcommand{\C}{{\mathbb C}}
\newcommand{\Z}{{\mathbb Z}}
\newcommand{\R}{{\mathbb R}}
\newcommand{\N}{{\mathbb N}}
\newcommand{\cP}{{\cal P}}
\newcommand\circlearound[1]{%
  \tikz[baseline]\node[draw,shape=circle,anchor=base] {#1} ;}



%%%%%%%%%%%%%%%%%%%%%%%%%%%%%%%%%%%%%%%%%

\begin{document}

\definecolor{darkgreen}{rgb}{0,0.6,0}
\definecolor{mylightlavender}{rgb}{0.9,0.9,1}
\definecolor{mylavender}{rgb}{0.4,0,1}
\definecolor{mygreen}{rgb}{0,0.6,0.4}
\definecolor{myorange}{rgb}{1,0.5,0} 


\pagestyle{empty}


Humboldt-Universit"at zu Berlin

Institut f"ur Mathematik

Unter den Linden 6

10099 Berlin\\[12mm]

\begin{center}
\textcolor{darkgreen}{\Huge{\textbf{19. Sommerschule}}}\\[6mm]

\textcolor{darkgreen}{\Huge{\textbf{Lust auf Mathematik}}}\\[12mm]


\Large{Berliner Netzwerk}

\Large{mathematisch-naturwissenschaftlich}

\Large{profilierter Schulen}\\[6mm]

\textcolor{darkgreen}{\Huge{\textbf{02. Juni bis 07. Juni 2019}}}\\[6mm]

\Large{im}

\Large{Jugendbildungszentrum}

\Large{Blossin e.V.}

\Large{Waldweg 10,
15754 Blossin}\\[12mm]



\end{center}
\pagebreak
\( \)
\pagebreak


\begin{center}

\vspace{-4cm}

%\includegraphics[width=130mm]{Fotos/alle.png}

\end{center}


\pagebreak




\tableofcontents
\[\begin{array}{c}
\\
\end{array}\]

\pagebreak
\[\begin{array}{rcl}
\\
\end{array}\]

\pagebreak
\pagestyle{plain}
\section*{Berichte der Gruppen}\addcontentsline{toc}{section}{Berichte der Gruppen}

\subsection*{Sph"arische Geometrie}\addcontentsline{toc}{subsection}{Sph"arische Geometrie}

\setcounter{Beispiel}{0}
\setcounter{equation}{0}

\setcounter{Satz}{0}
\setcounter{Lemma}{0}
\setcounter{Korollar}{0}
\setcounter{Definition}{0}
\setcounter{Bemerkung}{0}


\vspace{0,5cm}
\textit{Teilnehmer:}

\begin{tabular}{lp{3mm}l}
Luisa Aust&&Andreas-Gymnasium\\
Jan B"ottger&&Herder-Gymnasium\\
Murat Cecen&&Herder-Gymnasium\\
Ben Lennart Han"ske&&Immanuel-Kant-Gymnasium\\
Corban Hetscher&&Andreas-Gymnasium\\
Tabea Menkens&&Herder-Gymnasium\\
Franz Pfeiffer&&Andreas-Gymnasium\\
Eric Wallisch&&Andreas-Gymnasium\\
Torben Zepke&&Herder-Gymnasium\\[0,3cm]
\textit{mit tatkr"aftiger Unterst"utzung durch:}\\

Matthias Riddermann&& Humboldt-Universit\"at zu Berlin\\[0,3cm]

\textit{Gruppenleiter:}\\


Helga Baum&& Humboldt-Universit\"at zu Berlin\\
\end{tabular}
\begin{center}
%\includegraphics[width = \textwidth]{Fotos/Helga.png}
\end{center}

\pagebreak

\oursec{Erste "Uberschrift}

Bitte benutzt diese Befehle und Umgebungen und definiert keine eigenen Befehle. 

\begin{Beispiele}
Sch"one Beispiele
\end{Beispiele}

\oursubsec{Untergeordnete "Uberschrift}

\begin{Beispiel}

sch"ones Beispiel
\end{Beispiel}

\begin{Aufgabe}
interessante Aufgabe
\end{Aufgabe}

\begin{Ueberlegung}
wichtige "Uberlegung
\end{Ueberlegung}

\begin{Bezeichnung}
So wollen wir was bezeichnen...
\end{Bezeichnung}

\begin{Algorithmus}
Diesen Algorithmus benutzen wird...
\end{Algorithmus}
\begin{Ansatz}
Wir machen einen Ansatz
\end{Ansatz}

\begin{Behauptung}
Wir behaupten...
\end{Behauptung}

\begin{Satz}
Kluger Satz
\end{Satz}
\begin{Beweis}
... mit einem klugen Beweis
\end{Beweis}
\begin{Korollar}
... und einem klugen Korollar...
\end{Korollar}

\begin{Definition}
So definieren wir...
\end{Definition}
\begin{Bemerkung}
Das wollten wir schon immer mal feststellen...
\end{Bemerkung}
\begin{Lemma}
Ein nicht ganz so gro"ser Satz... Aber trotzdem wichtig...
\end{Lemma}

%%%%%%%%%%%%%%%%%%%%%%%%%%%%%%
%%%%%%%%%%%%%%%%%%%%%%%%%%%%%%%
%%%%%%%%%%%%%%%%%%%%%%%%%%%%%%

\newpage

\subsection*{Fraktale}\addcontentsline{toc}{subsection}{Fraktale}

\setcounter{Beispiel}{0}
\setcounter{equation}{0}

\setcounter{Satz}{0}
\setcounter{Lemma}{0}
\setcounter{Korollar}{0}
\setcounter{Definition}{0}
\setcounter{Bemerkung}{0}
\setcounter{oursec}{0}


\vspace{0,5cm}
\textit{Teilnehmer:}

\begin{tabular}{lp{3mm}l}
Dimitri Braun&&Heinrich-Hertz-Gymnasium\\
Daniel Busch&& Heinrich-Hertz-Gymnasium\\
Henrik Dickmann&&Heinrich-Hertz-Gymnasium\\
Lukas H"aring&&Heinrich-Hertz-Gymnasium\\
Miriahna Mahowski&&Andreas-Gymnasium\\
Robert Redlich&&Heinrich-Hertz-Gymnasium\\
Tita Rosemeyerr&&Heinrich-Hertz-Gymnasium\\
Friederike Stiller&&K"athe-Kollwitz-Gymnasium\\
Tan Mai Vu&& Heinrich-Hertz-Gymnasium\\
[0,3cm]

\textit{mit tatkr"aftiger Unterst"utzung durch:}\\

Christoph Werner&&Humboldt-Universit\"at zu Berlin\\[0,3cm]

\textit{Gruppenleiter:}\\


Andreas Filler&& Humboldt-Universit\"at zu Berlin\\

\end{tabular}
\begin{center}
%\includegraphics[width = \textwidth]{Fotos/Andreas.png}
\end{center}

\pagebreak


\oursec{Erste "Uberschrift}

Bitte benutzt diese Befehle und Umgebungen und definiert keine eigenen Befehle. 

\begin{Beispiele}
Sch"one Beispiele
\end{Beispiele}

\oursubsec{Untergeordnete "Uberschrift}

\begin{Beispiel}

sch"ones Beispiel
\end{Beispiel}

\begin{Aufgabe}
interessante Aufgabe
\end{Aufgabe}

\begin{Ueberlegung}
wichtige "Uberlegung
\end{Ueberlegung}

\begin{Bezeichnung}
So wollen wir was bezeichnen...
\end{Bezeichnung}

\begin{Algorithmus}
Diesen Algorithmus benutzen wird...
\end{Algorithmus}
\begin{Ansatz}
Wir machen einen Ansatz
\end{Ansatz}

\begin{Behauptung}
Wir behaupten...
\end{Behauptung}

\begin{Satz}
Kluger Satz
\end{Satz}
\begin{Beweis}
... mit einem klugen Beweis
\end{Beweis}
\begin{Korollar}
... und einem klugen Korollar...
\end{Korollar}

\begin{Definition}
So definieren wir...
\end{Definition}
\begin{Bemerkung}
Das wollten wir schon immer mal feststellen...
\end{Bemerkung}
\begin{Lemma}
Ein nicht ganz so gro"ser Satz... Aber trotzdem wichtig...
\end{Lemma}


%%%%%%%%%%%%%%%%%%%%%%%%%%%%%%%%%%
%%%%%%%%%%%%%%%%%%%%%%%%%%%%%%%%%%%%
%%%%%%%%%%%%%%%%%%%%%%%%%%%%%%%%%%
%
\pagebreak

\subsection*{Hyperbolische Geometrie}\addcontentsline{toc}{subsection}{Hyperbolische Geometrie}

\setcounter{Beispiel}{0}
\setcounter{equation}{0}

\setcounter{Satz}{0}
\setcounter{Lemma}{0}
\setcounter{Korollar}{0}
\setcounter{Definition}{0}
\setcounter{Bemerkung}{0}

\setcounter{oursec}{0}


\vspace{0,5cm}
\textit{Teilnehmer:}

\begin{tabular}{lp{3mm}l}
Samuel Bamrungbhuet&& Heinrich-Hertz-Gymnasium\\
Rahel Bensch&&Herder-Gymnasium\\
Matthias Bier&& Heinrich-Hertz-Gymnasium\\
Alec Dr"ucker&&Herder-Gymnasium\\
Ingmar Glauche&&Herder-Gymnasium\\
Marcal Herbrich&&Heinrich-Hertz-Gymnasium\\
Quynh Chau Nguyen&&Heinrich-Hertz-Gymnasium\\
Dennis Ossipov&&Heinrich-Hertz-Gymnasium\\
Mark Shafranov&&Herder-Gymnasium\\
[0,3cm]


\textit{mit tatkr"aftiger  Unterst"utzung durch:}\\
Luise Fehlinger && Humboldt-Universit\"at zu Berlin\\[0,3cm]

\textit{Gruppenleiter:}\\

Klaus Mohnke&& Humboldt-Universit\"at zu Berlin\\

\end{tabular}

\begin{center}
%\includegraphics[width = \textwidth]{Fotos/Klaus.png}
\end{center}

\pagebreak

\oursec{Erste "Uberschrift}

Bitte benutzt diese Befehle und Umgebungen und definiert keine eigenen Befehle. 

\begin{Beispiele}
Sch"one Beispiele
\end{Beispiele}

\oursubsec{Untergeordnete "Uberschrift}

\begin{Beispiel}

sch"ones Beispiel
\end{Beispiel}

\begin{Aufgabe}
interessante Aufgabe
\end{Aufgabe}

\begin{Ueberlegung}
wichtige "Uberlegung
\end{Ueberlegung}

\begin{Bezeichnung}
So wollen wir was bezeichnen...
\end{Bezeichnung}

\begin{Algorithmus}
Diesen Algorithmus benutzen wird...
\end{Algorithmus}
\begin{Ansatz}
Wir machen einen Ansatz
\end{Ansatz}

\begin{Behauptung}
Wir behaupten...
\end{Behauptung}

\begin{Satz}
Kluger Satz
\end{Satz}
\begin{Beweis}
... mit einem klugen Beweis
\end{Beweis}
\begin{Korollar}
... und einem klugen Korollar...
\end{Korollar}

\begin{Definition}
So definieren wir...
\end{Definition}
\begin{Bemerkung}
Das wollten wir schon immer mal feststellen...
\end{Bemerkung}
\begin{Lemma}
Ein nicht ganz so gro"ser Satz... Aber trotzdem wichtig...
\end{Lemma}

 
\newpage
%%%%%%%%%%%%%%%%%%%%%%%%%%%%%%%%%%%
%%%%%%%%%%%%%%%%%%%%%%%%%%%%%%%%%%%%
%%%%%%%%%%%%%%%%%%%%%%%%%%%%%%%%%%%

\subsection*{Zuf"allig oder signifikant?}\addcontentsline{toc}{subsection}{Zuf"allig oder signifikant?}

\setcounter{Beispiel}{0}
\setcounter{equation}{0}

\setcounter{Satz}{0}
\setcounter{Lemma}{0}
\setcounter{Korollar}{0}
\setcounter{Definition}{0}
\setcounter{Bemerkung}{0}

\setcounter{oursec}{0}


\vspace{0,5cm}
\textit{Teilnehmer:}

\begin{tabular}{lp{3mm}l}
Aram Azarvash&&Herder-Gymnasium\\
Ibo Becker&&K"athe-Kollwitz-Gymnasium\\
Johanna Bellon&&Heinrich-Hertz-Gymnasium\\
Anna Julia Hauschild&&Immanuel-Kant-Gymnasium\\
Dennis Jurk&&Herder-Gymnasium\\
Laura K"alberer&&Herder-Gymnasium\\
Leonhard Klann&&Immanuel-Kant-Gymnasium\\
Yassin Ouali&&Herder-Gymnasium\\[0,3cm]
\textit{mit tatkr"aftiger  Unterst"utzung durch:}\\


Niklas Sturm && Humboldt-Universit"at zu Berlin\\[0,3cm]


\textit{Gruppenleiter:}\\

Markus Rei"s&&Humboldt-Universit\"at zu Berlin\\
Randolf Altmeyer&& Humboldt-Universit"at zu Berlin
\end{tabular}


\begin{center}

%\includegraphics[width = \textwidth]{Fotos/Markus-Randolf.png}

\end{center}

\pagebreak


\oursec{Erste "Uberschrift}

Bitte benutzt diese Befehle und Umgebungen und definiert keine eigenen Befehle. 

\begin{Beispiele}
Sch"one Beispiele
\end{Beispiele}

\oursubsec{Untergeordnete "Uberschrift}

\begin{Beispiel}

sch"ones Beispiel
\end{Beispiel}

\begin{Aufgabe}
interessante Aufgabe
\end{Aufgabe}

\begin{Ueberlegung}
wichtige "Uberlegung
\end{Ueberlegung}

\begin{Bezeichnung}
So wollen wir was bezeichnen...
\end{Bezeichnung}

\begin{Algorithmus}
Diesen Algorithmus benutzen wird...
\end{Algorithmus}
\begin{Ansatz}
Wir machen einen Ansatz
\end{Ansatz}

\begin{Behauptung}
Wir behaupten...
\end{Behauptung}

\begin{Satz}
Kluger Satz
\end{Satz}
\begin{Beweis}
... mit einem klugen Beweis
\end{Beweis}
\begin{Korollar}
... und einem klugen Korollar...
\end{Korollar}

\begin{Definition}
So definieren wir...
\end{Definition}
\begin{Bemerkung}
Das wollten wir schon immer mal feststellen...
\end{Bemerkung}
\begin{Lemma}
Ein nicht ganz so gro"ser Satz... Aber trotzdem wichtig...
\end{Lemma}

%%%%%%%%%%%%%%%%%%%%%%%%%%%%%%
%%%%%%%%%%%%%%%%%%%%%%%%%%%%%%%
%%%%%%%%%%%%%%%%%%%%%%%%%%%%%%

\newpage
\subsection*{Vom Pendel "ubers Integral und Python zum L"otkolben und zur"ck}\addcontentsline{toc}{subsection}{Vom Pendel "ubers Integral und Python zum L"otkolben und zur"ck}

\setcounter{Beispiel}{0}
\setcounter{equation}{0}

\setcounter{Satz}{0}
\setcounter{Lemma}{0}
\setcounter{Korollar}{0}
\setcounter{Definition}{0}
\setcounter{Bemerkung}{0}
\setcounter{oursec}{0}

\vspace{0,5cm}
\textit{Teilnehmer:}

\begin{tabular}{lp{3mm}l}
Nadin Adham&&Herder-Gymnasium\\
Florian Brandt&&K"athe-Kollwitz-Gymnasium\\
Richard Fuchs&&Herder-Gymnasium\\
Anselm Herkstr"oter&&Herder-Gymnasium\\
Leonard Hiesgen&&Andreas-Gymnasium\\
Conni M"arz&&Heinrich-Hertz-Gymnasium\\
Malte Pasternak&&K"athe-Kollwitz-Gymnasium\\
Jean-Pasqual Sindermann&Herder-Gymnasium\\
Martin Wundermann&&K"athe-Kollwitz-Gymnasium\\[0,3cm]


\textit{Gruppenleiter:}\\

Claus F"uhrer&&Lund University\\
Ren\'{e} Lamour&&Humboldt-Universit"at zu Berlin\\
Caren Tischendorf&&Humboldt-Universit"at zu Berlin, MATH+

\end{tabular}
\begin{center}
%\includegraphics[width = \textwidth]{Fotos/Lucas.png}
\end{center}

\pagebreak


\oursec{Erste "Uberschrift}

Bitte benutzt diese Befehle und Umgebungen und definiert keine eigenen Befehle. 

\begin{Beispiele}
Sch"one Beispiele
\end{Beispiele}

\oursubsec{Untergeordnete "Uberschrift}

\begin{Beispiel}

sch"ones Beispiel
\end{Beispiel}

\begin{Aufgabe}
interessante Aufgabe
\end{Aufgabe}

\begin{Ueberlegung}
wichtige "Uberlegung
\end{Ueberlegung}

\begin{Bezeichnung}
So wollen wir was bezeichnen...
\end{Bezeichnung}

\begin{Algorithmus}
Diesen Algorithmus benutzen wird...
\end{Algorithmus}
\begin{Ansatz}
Wir machen einen Ansatz
\end{Ansatz}

\begin{Behauptung}
Wir behaupten...
\end{Behauptung}

\begin{Satz}
Kluger Satz
\end{Satz}
\begin{Beweis}
... mit einem klugen Beweis
\end{Beweis}
\begin{Korollar}
... und einem klugen Korollar...
\end{Korollar}

\begin{Definition}
So definieren wir...
\end{Definition}
\begin{Bemerkung}
Das wollten wir schon immer mal feststellen...
\end{Bemerkung}
\begin{Lemma}
Ein nicht ganz so gro"ser Satz... Aber trotzdem wichtig...
\end{Lemma}

%%%%%%%%%%%%%%%%%%%%%%%%%%%%%%
%%%%%%%%%%%%%%%%%%%%%%%%%%%%%%%
%%%%%%%%%%%%%%%%%%%%%%%%%%%%%%

\newpage
\subsection*{Die Euler'sche Summenformel}\addcontentsline{toc}{subsection}{{Sichere Kommunikation}Die Euler'sche Summenformel}

\setcounter{Beispiel}{0}
\setcounter{equation}{0}

\setcounter{Satz}{0}
\setcounter{Lemma}{0}
\setcounter{Korollar}{0}
\setcounter{Definition}{0}
\setcounter{Bemerkung}{0}
\setcounter{oursec}{0}

\vspace{0,5cm}
\textit{Teilnehmer:}

\begin{tabular}{lp{3mm}l}

Deniz Coskun&&K"athe-Kollwitz-Gymnasium\\
Paul Cronin&&Heinrich-Hertz-Gymnasium\\
Theodor Funke&&Herder-Gymnasium\\
Emilia Grafe&&Andreas-Gymnasium\\
Nikolas Kilian&&Heinrich-Hertz-Gymnasium\\
Oona Kintscher&&K"athe-Kollwitz-Gymnasium\\
Teresa L"offelhardt&&K"athe-Kollwitz-Gymnasium\\
Mattan Schremmer&&K"athe-Kollwitz-Gymnasium\\
Paul Siewert&&Heinrich-Hertz-Gymnasium\\[0,3cm]
\textit{mit tatkr"aftiger Unterst"utzung durch:}\\
Tobias Kretzschmar&&Humboldt-Universit\"at zu Berlin\\[0,3cm]
\textit{Gruppenleiter:}\\

Eren Ucar&&Humboldt-Universit"at zu Berlin

\end{tabular}
\begin{center}
%\includegraphics[width = \textwidth]{Fotos/Eren.png}
\end{center}

\pagebreak

\oursec{Erste "Uberschrift}

Bitte benutzt diese Befehle und Umgebungen und definiert keine eigenen Befehle. 

\begin{Beispiele}
Sch"one Beispiele
\end{Beispiele}

\oursubsec{Untergeordnete "Uberschrift}

\begin{Beispiel}

sch"ones Beispiel
\end{Beispiel}

\begin{Aufgabe}
interessante Aufgabe
\end{Aufgabe}

\begin{Ueberlegung}
wichtige "Uberlegung
\end{Ueberlegung}

\begin{Bezeichnung}
So wollen wir was bezeichnen...
\end{Bezeichnung}

\begin{Algorithmus}
Diesen Algorithmus benutzen wird...
\end{Algorithmus}
\begin{Ansatz}
Wir machen einen Ansatz
\end{Ansatz}

\begin{Behauptung}
Wir behaupten...
\end{Behauptung}

\begin{Satz}
Kluger Satz
\end{Satz}
\begin{Beweis}
... mit einem klugen Beweis
\end{Beweis}
\begin{Korollar}
... und einem klugen Korollar...
\end{Korollar}

\begin{Definition}
So definieren wir...
\end{Definition}
\begin{Bemerkung}
Das wollten wir schon immer mal feststellen...
\end{Bemerkung}
\begin{Lemma}
Ein nicht ganz so gro"ser Satz... Aber trotzdem wichtig...
\end{Lemma}


\pagebreak


\section*{Auswertung der Befragung}\addcontentsline{toc}{section}{Auswertung der Befragung}

\end{document}