\documentclass[12pt]{article}
\usepackage[a4paper,margin=2.5cm,footskip=0.5cm]{geometry}
\usepackage{amsmath}
\newcommand{\overtext}[2]{\mathrel{\overset{\makebox[0pt]{\mbox{\normalfont\tiny\sffamily #2}}}{#1}}}
\usepackage{mathtools}
\usepackage{scalerel,amssymb}
\usepackage{amsthm}
\usepackage{tikz,pgfplots}
\newcommand{\comment}[1]{}
\DeclarePairedDelimiter\abs{\lvert}{\rvert}
\begin{document}
\title{\vspace{-2.0cm}Sommerschule}
\author{Nikolas Kilian
Projektleiter: Eren Ucar}
\date{\today}
\maketitle

\section{Brenoulli Zahlen und Polynome}
\subsection{Bernoulli Zahlen}
\subsubsection{Definition}

Die Bernoulli Zahlen \(B_n, n \in \mathbb{N}_0\), bilden eine rationale Zahlenfolge.
Sie sind rekursiv wie folgt definiert.
\[B_0 := 1\]
\[\sum_{k=0}^n {n+1 \choose k} \cdot B_k = 0, n \in \mathbb{N}\]

\subsection{Bernoulli Polynome}
\subsubsection{Definition}

Die Bernoulli Polynome \(B_n:\mathbb{R}\to\mathbb{R},n\in\mathbb{N}_0\), sind wie folt definiert:
\[B_n(x) := \sum_{k=0}^n {n \choose k} \cdot B_k \cdot x^{n-k}, x\in\mathbb{R}\]

\[n = 0: B_0(x) = 1\]
\[n = 1: B_1(x) = x - \frac{1}{2}\]
\[n = 2: B_2(x) = x^2 - x + \frac{1}{6}\]

\subsubsection{Eigenschaften}
\paragraph{E1} 
\[B_n(0) = B_n(1), \forall n\in\mathbb{N}, n \neq 1\]

\begin{proof}[Beweis]
\[B_n(0) = B_n(1), \forall n\in\mathbb{N}, n \neq 1\]
\[\sum_{k=0}^n {n \choose k} \cdot B_k \cdot 0^{n-k} = \sum_{k=0}^n {n \choose k} \cdot B_k \cdot 1^{n-k}\]
\[\sum_{k=0}^n {n \choose k} \cdot B_k \cdot 0 = \sum_{k=0}^n {n \choose k} \cdot B_k \cdot 1\]
\[0 = \sum_{k=0}^n {n \choose k} \cdot B_k\]
Gilt laut Definition.
\end{proof}

\paragraph{E2} 
\[\frac{d}{dx} B_n(x) = n \cdot B_{n-1}(x), \forall n\in\mathbb{N}, n \geq 1\]

\begin{proof}[Beweis]
\[\frac{d}{dx} B_n(x) = n \cdot B_{n-1}(x), \forall n\in\mathbb{N}, n \geq 1\]
\[\frac{d}{dx} \sum_{k=0}^n {n \choose k} \cdot B_k \cdot x^{n-k} = n \cdot \sum_{k=0}^{n-1} {n-1 \choose k} \cdot B_k \cdot x^{n-1-k}\]
\[\sum_{k=0}^n {n \choose k} \cdot B_k \cdot \frac{d}{dx} x^{n-k} = \sum_{k=0}^{n-1} n \cdot {n-1 \choose k} \cdot B_k \cdot x^{n-1-k}\]
Da der Terrm für \(k=n\) konstant ist, ist dessen Ableitung 0. Es kann demnach aus der Summenformel ausgelassen werden.
\[\sum_{k=0}^{n-1} {n \choose k} \cdot B_k \cdot (n-k) \cdot x^{n-k-1} = \sum_{k=0}^{n-1} n \cdot {n-1 \choose k} \cdot B_k \cdot x^{n-1-k}\]
Der nächste Schritt wird dem unüberzeugtem Leser als Übung überlassen.
\end{proof}

\[\Longrightarrow \int B_{n-1}(x) dx = \frac{1}{n} B_n(x)\]

\paragraph{E3} 
\[\int_0^1 B_n(x) dx = 0, \forall n\in\mathbb{N}, n \geq 1\]

\begin{proof}[Beweis]
\[\int_0^1 B_n(x) dx = 0, \forall n\in\mathbb{N}, n \geq 1\]
\[\int_0^1 B_n(x) dx \overtext{=}{(E2)} \left. \frac{1}{n+1} B_{n+1}(x) \right|_0^1 \overtext{=}{(E1)} 0\]
\end{proof}

\subsection{Alternative Definition}

Die Bernoulli Polynome kann man über folgender Funktionalgleichungen charakterisieren.

\newtheorem{theorem}{Theorem}[section]
\begin{theorem}[Theorem]
Für \(n \in \mathbb{N}, x \in \mathbb{R}\) gilt:

\[F_n(x) = B_n(x+1) - B_n(x) = nx^{n-1}\]
\end{theorem}

\[B_n(x+1) - B_n(x) = nx^{n-1}\]
\begin{proof}
%\paragraph{Induktionsanfang}
Sei \(n = 0\).
Es gilt \[F_0(x) = B_0(x+1) - B_0(x) = 0x^{0-1}\]
\begin{proof}
\[F_0(x) = B_0(x+1) - B_0(x) = 0x^{0-1}\]
\[F_0(x) = \sum_{k=0}^0 {0 \choose k} \cdot B_k \cdot x^{0-k} - \sum_{k=0}^0 {0 \choose k} \cdot B_k \cdot x^{0-k} = 0x^{0-1}\]
\[F_0(x) = B_0 \cdot x^{-k} - B_0 \cdot x^{-k} = 0x^{0-1}\]
\[F_0(x) = 1 - 1 = 0\]
\end{proof}

%\paragraph{Induktionsbehaupptung}
Sei \(n \in \mathbb{N}, n \geq 0\) beliebig aber fest gewählt.
\[B_n(x+1) - B_n(x) = nx^{n-1} \Longrightarrow B_{n+1}(x+1) - B_{n+1}(x) = (n+1) \cdot x^n\]

%\paragraph{Induktionsschritt}
\begin{proof}
\[B_{n+1}(x+1) - B_{n+1}(x) = (n+1) \cdot x^n\]
\[\frac{d}{dx} B_{n+1}(x+1) - \frac{d}{dx} B_{n+1}(x) = \frac{d}{dx} (n+1) \cdot x^n\]
\[(n+1) \cdot B_{n+1-1}(x+1) - (n+1) \cdot B_{n+1-1}(x) = (n+1) \cdot nx^{n-1}\]
\[B_{n}(x+1) - B_{n}(x) = nx^{n-1}\]
\end{proof}

Da \(n\) beliebig aber fest gewählt ist, gilt die Behauptung für alle \(n \in \mathbb{N}, n \geq 0\).
\end{proof}

\newtheorem{corollary}{Corollary}[theorem]
\begin{corollary}
\[B_{2m+1}=0, \forall m \in \mathbb{N}\]
\end{corollary}
\begin{proof}
%TODO
Sei \(n\) gerade.
\[B_{n+1}(x+1) - B_{n+1}(x) = (n+1) \cdot x^n\]
Sei \(x := -\frac{1}{2}\)
\[B_{n+1}\left(1-\frac{1}{2}\right) - B_{n+1}\left(-\frac{1}{2}\right) = (n+1)\cdot\left(-\frac{1}{2}\right)^n\]
\[B_{n+1}\left(\frac{1}{2}\right) - B_{n+1}\left(-\frac{1}{2}\right) = (n+1)\cdot\left(-\frac{1}{2}\right)^n\]
\[\sum_{k=0}^{n+1} {n+1 \choose k} \cdot B_k \cdot \left(\frac{1}{2}\right)^{n+1-k} - \sum_{k=0}^{n+1} {n+1 \choose k} \cdot B_k \cdot \left(-\frac{1}{2}\right)^{n+1-k} = (n+1)\cdot\left(-\frac{1}{2}\right)^n\]
\[\sum_{k=0}^{n+1} {n+1 \choose k} \cdot B_k \cdot \left(\left(\frac{1}{2}\right)^{n+1-k} - \left(-\frac{1}{2}\right)^{n+1-k}\right) = (n+1)\cdot\left(-\frac{1}{2}\right)^n\]
Wenn \(n + 1 - k\) gerade ist, heben sich die Summanden auf. Da \(n\) gerade ist, ist \(n + 1 - k\) gerade genau dann, wenn \(1-k\) gerade ist, also wenn \(k\) ungerade ist.

\end{proof}

\begin{corollary}
\[\sum_{k=1}^N k^n = \frac{B_{n+1}(N+1)-B_{n+1}}{n+1}, n, N \in \mathbb{N}\]
\end{corollary}
\begin{proof}
%TODO
\[\sum_{k=1}^N k^n = \frac{B_{n+1}(N+1) - B_{n+1}}{n+1}\]
\[\sum_{k=1}^N (n+1) \cdot k^n = B_{n+1}(N+1) - B_{n+1}\]
\[\sum_{k=1}^N B_{n+1}(k+1) - B_{n+1}(k) = B_{n+1}(N+1) - B_{n+1}\]
\[B_{n+1}(N+1) - B_{n+1}(1) = B_{n+1}(N+1) - B_{n+1}\]
\[B_{n+1}(N+1) - \sum_{k=0}^{n+1} {n+1 \choose k} \cdot B_k \cdot 1^{n+1-k} = B_{n+1}(N+1) - B_{n+1}\]
\[B_{n+1}(N+1) - {n+1 \choose n+1} \cdot B_{n+1} - \sum_{k=0}^{n} {n+1 \choose k} \cdot B_k = B_{n+1}(N+1) - B_{n+1}\]
Laut Definition der Bernoulli Zahlen gilt:
\[B_{n+1}(N+1) - 1 \cdot B_{n+1} - 0 = B_{n+1}(N+1) - B_{n+1}\]
\[B_{n+1}(N+1) - B_{n+1} = B_{n+1}(N+1) - B_{n+1}\]
\end{proof}

\section{Euler'sche Summenformel}

Wir setzen die Bernoulli Polynome auf \([0,1)\) periodisch fort auf \(\mathbb{R}\)

\subsection{Definition}

\[P_n(x) := B_n(x-[x])\]
\[[x] := \max \{ k \in \mathbb{Z} | k \leq x \} \]

\begin{theorem}{Einfache Euler'sche Summenformel}
Sei \(n\in\mathbb{N}\) beliebig und \(f : [0, n] \to \mathbb{R}\) eine stetig differenzierbare Funktion.
\[\sum_{k=0}^n f(k) = \int_0^n f(x) dx + \frac{1}{2} \cdot (f(0) + f(n)) + R\]
wobei \(R_1 := \int_0^n f'(x) \cdot P_1(x) dx\)
\end{theorem}
\begin{proof}
Sei \(k \in \mathbb{N}, x \in [k, k+1) \Longrightarrow P'_1(x) = P_0(x) = 1\)
Für \(k \in \{ 0, 1, ..., n-1 \}\):

\[\int_k^{k+1} f(x) dx = \int_k^{k+1} f(x) P'_1(x) dx = \lim_{\epsilon \downarrow 0} \int_k^{k+1-\epsilon} F(x) P'_1(x) dx\]
\[= \lim_{\epsilon \downarrow 0} \left(\left. f(x) \cdot P_1(x) \right|_{x=k}^{k+1-\epsilon} - \int_k^{k+1-\epsilon} f'(x) P_1(x) dx \right) \] %TODO
\[\lim_{\epsilon \downarrow 0} \left(f(k+1-\epsilon)P_1(k+1-\epsilon) -  f(k)P_1(k) - \int_k^{k+1-\epsilon} f'(x) P_1(x) dx \right)\]
\[ = \frac{1}{2}(f(k+1)+f(k)) - \int_k^{k+1} f'(x) P_1(x) dx\]

\[\int_k^{k+1} f(x) dx = \frac{1}{2}(f(k+1)+f(k)) - \int_k^{k+1} f'(x) P_1(x) dx\]

Summieren wir die obige Gleichung für \(k = 0, 1, ..., n-1\)

\[\int_0^n f(x) dx = \frac{1}{2} \sum_{k=0}^{n-1} \left(f(k+1) + f(k)\right) - \int_0^n f'(x) P_1(x) dx\]
\[= \frac{1}{2}(f(0) + f(n)) + \sum_{k=1}^{n-1} f(k) - \int_0^n f'(x) P_1(x) dx\]

Addiere auf beiden Seiten \(\frac{1}{2}(f(0) + f(n))\) und stelle um. Dann folgt die Behauptung.
\end{proof}

\begin{theorem}{Allgemeine Euler'sche Summenformel}
Sei \(n \in \mathbb{N}\) und \(f : [0, n] \to \mathbb{R}\) sei n-mal stetig differenzierbar, \(m \geq 2\). Dann gilt:

\[\sum_{k=0}^n f(k) = \int_0^n f(x) dx + \frac{1}{2}(f(0)+f(n)) + \sum_{k=2}^m \left. \frac{(-1)^k B_k}{k!} f^{(k-1)} (x) \right|_{x=0}^n + R_m\]

mit \(R_m := \frac{(-1)^{m+1}}{m!} \int_0^n f^{(m)}(x) P_m(x) dx\)
\end{theorem}
\begin{proof}
Wegen Satz 1 reicht es zu zeigen \(\int_0^n f'(x) P_1(x) dx = \sum_{k=2}^m \left. \frac{(-1)^k B_k}{k!} f^{(k-1)}(x) \right|_{x=0}^n + R_m\)

\[\int_0^n f'(x) P_1(x) dx = \sum_{k=2}^m \left. \frac{(-1)^k B_k}{k!} f^{(k-1)}(x) \right|_{x=0}^n + R_m\]
\[\int_0^n f'(x) P_1(x) dx = \left. \frac{1}{2!} f'(x) P_2(x) \right|_{x=0}^n + \left. \frac{1}{3!} f''(x) P_3(x) \right|_{x=0}^n + ...\]\[ + \left. \frac{1}{(m)!} f^{(m-1)}(x) P_{m}(x) \right|_{x=0}^n + \frac{(-1)^{m+1}}{m!} \int_0^n f^{(m)}(x) P_m(x) dx \]
\[ = \sum_{k=2}^m \frac{P_{k}(x)}{k!} f^{(k-1)}(x) + \frac{(-1)^{m+1}}{m!} \int_0^n f^{(m)}(x) P_m(x) dx \]
\end{proof}

\end{document}