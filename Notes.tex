\documentclass[12pt]{article}
\usepackage[a4paper,margin=2.5cm,footskip=0.5cm]{geometry}
\usepackage{amsmath}
\newcommand{\overtext}[2]{\mathrel{\overset{\makebox[0pt]{\mbox{\normalfont\tiny\sffamily #2}}}{#1}}}
\usepackage{mathtools}
\usepackage{scalerel,amssymb}
\usepackage{amsthm}
\usepackage{tikz}
\newcommand{\comment}[1]{}
\DeclarePairedDelimiter\abs{\lvert}{\rvert}
\begin{document}
\title{\vspace{-2.0cm}Sommerschule}
\author{Nikolas Kilian
Projektleiter: Eren Ucar}
\date{\today}
\maketitle

\tableofcontents

\section{Brenoulli Zahlen und Polynome}
\subsection{Bernoulli Zahlen}
\subsubsection{Definition}

Die Bernoulli Zahlen \(B_n, n \in \mathbb{N}_0\), bilden eine rationale Zahlenfolge.
Sie sind rekursiv wie folgt definiert.
\[B_0 := 1\]
\[\sum_{k=0}^n {n+1 \choose k} \cdot B_k = 0, n \in \mathbb{N}\]

\subsection{Bernoulli Polynome}
\subsubsection{Definition}

Die Bernoulli Polynome \(B_n:\mathbb{R}\to\mathbb{R},n\in\mathbb{N}_0\), sind wie folt definiert:
\[B_n(x) := \sum_{k=0}^n {n \choose k} \cdot B_k \cdot x^{n-k}, x\in\mathbb{R}\]

\[n = 0: B_0(x) = 1\]
\[n = 1: B_1(x) = x - \frac{1}{2}\]
\[n = 2: B_2(x) = x^2 - x + \frac{1}{6}\]

\subsubsection{Eigenschaften}
\paragraph{E1} 
\[B_n(0) = B_n(1), \forall n\in\mathbb{N}, n \neq 1\]

\begin{proof}[Beweis]
\[B_n(0) = B_n(1), \forall n\in\mathbb{N}, n \neq 1\]
\[\sum_{k=0}^n {n \choose k} \cdot B_k \cdot 0^{n-k} = \sum_{k=0}^n {n \choose k} \cdot B_k \cdot 1^{n-k}\]
\[\sum_{k=0}^n {n \choose k} \cdot B_k \cdot 0 = \sum_{k=0}^n {n \choose k} \cdot B_k \cdot 1\]
\[0 = \sum_{k=0}^n {n \choose k} \cdot B_k\]
Gilt laut Definition.
\end{proof}

\paragraph{E2} 
\[\frac{d}{dx} B_n(x) = n \cdot B_{n-1}(x), \forall n\in\mathbb{N}, n \geq 1\]

\begin{proof}[Beweis]
\[\frac{d}{dx} B_n(x) = n \cdot B_{n-1}(x), \forall n\in\mathbb{N}, n \geq 1\]
\[\frac{d}{dx} \sum_{k=0}^n {n \choose k} \cdot B_k \cdot x^{n-k} = n \cdot \sum_{k=0}^{n-1} {n-1 \choose k} \cdot B_k \cdot x^{n-1-k}\]
\[\sum_{k=0}^n {n \choose k} \cdot B_k \cdot \frac{d}{dx} x^{n-k} = \sum_{k=0}^{n-1} n \cdot {n-1 \choose k} \cdot B_k \cdot x^{n-1-k}\]
Da der Terrm für \(k=n\) konstant ist, ist dessen Ableitung 0. Es kann demnach aus der Summenformel ausgelassen werden.
\[\sum_{k=0}^{n-1} {n \choose k} \cdot B_k \cdot (n-k) \cdot x^{n-k-1} = \sum_{k=0}^{n-1} n \cdot {n-1 \choose k} \cdot B_k \cdot x^{n-1-k}\]
Der nächste Schritt wird dem unüberzeugtem Leser als Übung überlassen.
\end{proof}

\[\Longrightarrow \int B_{n-1}(x) dx = \frac{1}{n} B_n(x)\]

\paragraph{E3} 
\[\int_0^1 B_n(x) dx = 0, \forall n\in\mathbb{N}, n \geq 1\]

\begin{proof}[Beweis]
\[\int_0^1 B_n(x) dx = 0, \forall n\in\mathbb{N}, n \geq 1\]
\[\int_0^1 B_n(x) dx \overtext{=}{(E2)} \left. \frac{1}{n+1} B_{n+1}(x) \right|_0^1 \overtext{=}{(E1)} 0\]
\end{proof}

\subsection{Alternative Definition}

Die Bernoulli Polynome kann man über folgender Funktionalgleichungen charakterisieren.

\newtheorem{theorem}{Theorem}[section]
\begin{theorem}[Theorem]
Für \(n \in \mathbb{N}, x \in \mathbb{R}\) gilt:

\[F_n(x) = B_n(x+1) - B_n(x) = nx^{n-1}\]
\end{theorem}

\[B_n(x+1) - B_n(x) = nx^{n-1}\]
\begin{proof}
%\paragraph{Induktionsanfang}
Sei \(n = 0\).
Es gilt \[F_0(x) = B_0(x+1) - B_0(x) = 0x^{0-1}\]
\begin{proof}
\[F_0(x) = B_0(x+1) - B_0(x) = 0x^{0-1}\]
\[F_0(x) = \sum_{k=0}^0 {0 \choose k} \cdot B_k \cdot x^{0-k} - \sum_{k=0}^0 {0 \choose k} \cdot B_k \cdot x^{0-k} = 0x^{0-1}\]
\[F_0(x) = B_0 \cdot x^{-k} - B_0 \cdot x^{-k} = 0x^{0-1}\]
\[F_0(x) = 1 - 1 = 0\]
\end{proof}

%\paragraph{Induktionsbehaupptung}
Sei \(n \in \mathbb{N}, n \geq 0\) beliebig aber fest gewählt.
\[B_n(x+1) - B_n(x) = nx^{n-1} \Longrightarrow B_{n+1}(x+1) - B_{n+1}(x) = (n+1) \cdot x^n\]

%\paragraph{Induktionsschritt}
\begin{proof}
\[B_{n+1}(x+1) - B_{n+1}(x) = (n+1) \cdot x^n\]
\[\frac{d}{dx} B_{n+1}(x+1) - \frac{d}{dx} B_{n+1}(x) = \frac{d}{dx} (n+1) \cdot x^n\]
\[(n+1) \cdot B_{n+1-1}(x+1) - (n+1) \cdot B_{n+1-1}(x) = (n+1) \cdot nx^{n-1}\]
\[B_{n}(x+1) - B_{n}(x) = nx^{n-1}\]
\end{proof}

Da \(n\) beliebig aber fest gewählt ist, gilt die Behauptung für alle \(n \in \mathbb{N}, n \geq 0\).
\end{proof}

\newtheorem{corollary}{Corollary}[theorem]
\begin{corollary}
\[B_{2m+1}=0, \forall m \in \mathbb{N}\]
\end{corollary}
\begin{proof}
%TODO
Sei \(n\) gerade.
\[B_{n+1}(x+1) - B_{n+1}(x) = (n+1) \cdot x^n\]
Sei \(x := -\frac{1}{2}\)
\[B_{n+1}\left(1-\frac{1}{2}\right) - B_{n+1}\left(-\frac{1}{2}\right) = (n+1)\cdot\left(-\frac{1}{2}\right)^n\]
\[B_{n+1}\left(\frac{1}{2}\right) - B_{n+1}\left(-\frac{1}{2}\right) = (n+1)\cdot\left(-\frac{1}{2}\right)^n\]
\[\sum_{k=0}^{n+1} {n+1 \choose k} \cdot B_k \cdot \left(\frac{1}{2}\right)^{n+1-k} - \sum_{k=0}^{n+1} {n+1 \choose k} \cdot B_k \cdot \left(-\frac{1}{2}\right)^{n+1-k} = (n+1)\cdot\left(-\frac{1}{2}\right)^n\]
\[\sum_{k=0}^{n+1} {n+1 \choose k} \cdot B_k \cdot \left(\left(\frac{1}{2}\right)^{n+1-k} - \left(-\frac{1}{2}\right)^{n+1-k}\right) = (n+1)\cdot\left(-\frac{1}{2}\right)^n\]
Wenn \(n + 1 - k\) gerade ist, heben sich die Summanden auf. Da \(n\) gerade ist, ist \(n + 1 - k\) gerade genau dann, wenn \(1-k\) gerade ist, also wenn \(k\) ungerade ist.

\end{proof}

\begin{corollary}
\[\sum_{k=1}^N k^n = \frac{B_{n+1}(N+1)-B_{n+1}}{n+1}, n, N \in \mathbb{N}\]
\end{corollary}
\begin{proof}
%TODO
\[\sum_{k=1}^N k^n = \frac{B_{n+1}(N+1) - B_{n+1}}{n+1}\]
\[\sum_{k=1}^N (n+1) \cdot k^n = B_{n+1}(N+1) - B_{n+1}\]
\[\sum_{k=1}^N B_{n+1}(k+1) - B_{n+1}(k) = B_{n+1}(N+1) - B_{n+1}\]
\[B_{n+1}(N+1) - B_{n+1}(1) = B_{n+1}(N+1) - B_{n+1}\]
\[B_{n+1}(N+1) - \sum_{k=0}^{n+1} {n+1 \choose k} \cdot B_k \cdot 1^{n+1-k} = B_{n+1}(N+1) - B_{n+1}\]
\[B_{n+1}(N+1) - {n+1 \choose n+1} \cdot B_{n+1} - \sum_{k=0}^{n} {n+1 \choose k} \cdot B_k = B_{n+1}(N+1) - B_{n+1}\]
Laut Definition der Bernoulli Zahlen gilt:
\[B_{n+1}(N+1) - 1 \cdot B_{n+1} - 0 = B_{n+1}(N+1) - B_{n+1}\]
\[B_{n+1}(N+1) - B_{n+1} = B_{n+1}(N+1) - B_{n+1}\]
\end{proof}

\section{Euler'sche Summenformel}

Wir setzen die Bernoulli Polynome auf \([0,1)\) periodisch fort auf \(\mathbb{R}\)

\subsection{Definition}

\[P_n(x) := B_n(x-[x])\]
\[[x] := \max \{ k \in \mathbb{Z} | k \leq x \} \]

\begin{theorem}{Einfache Euler'sche Summenformel}
Sei \(n\in\mathbb{N}\) beliebig und \(f : [0, n] \to \mathbb{R}\) eine stetig differenzierbare Funktion.
\[\sum_{k=0}^n f(k) = \int_0^n f(x) dx + \frac{1}{2} \cdot (f(0) + f(n)) + R_1\]
wobei \(R_1 := \int_0^n f'(x) \cdot P_1(x) dx\)
\end{theorem}
\begin{proof}
Sei \(k \in \mathbb{N}, x \in [k, k+1) \Longrightarrow P'_1(x) = P_0(x) = 1\)
Für \(k \in \{ 0, 1, ..., n-1 \}\):

\[\int_k^{k+1} f(x) dx = \int_k^{k+1} f(x) P'_1(x) dx = \lim_{\epsilon \downarrow 0} \int_k^{k+1-\epsilon} F(x) P'_1(x) dx\]
\[= \lim_{\epsilon \downarrow 0} \left(\left. f(x) \cdot P_1(x) \right|_{x=k}^{k+1-\epsilon} - \int_k^{k+1-\epsilon} f'(x) P_1(x) dx \right) \] %TODO
\[\lim_{\epsilon \downarrow 0} \left(f(k+1-\epsilon)P_1(k+1-\epsilon) -  f(k)P_1(k) - \int_k^{k+1-\epsilon} f'(x) P_1(x) dx \right)\]
\[ = \frac{1}{2}(f(k+1)+f(k)) - \int_k^{k+1} f'(x) P_1(x) dx\]

\[\int_k^{k+1} f(x) dx = \frac{1}{2}(f(k+1)+f(k)) - \int_k^{k+1} f'(x) P_1(x) dx\]

Summieren wir die obige Gleichung für \(k = 0, 1, ..., n-1\)

\[\int_0^n f(x) dx = \frac{1}{2} \sum_{k=0}^{n-1} \left(f(k+1) + f(k)\right) - \int_0^n f'(x) P_1(x) dx\]
\[= \frac{1}{2}(f(0) + f(n)) + \sum_{k=1}^{n-1} f(k) - \int_0^n f'(x) P_1(x) dx\]

Addiere auf beiden Seiten \(\frac{1}{2}(f(0) + f(n))\) und stelle um. Dann folgt die Behauptung.
\end{proof}

\begin{theorem}{Allgemeine Euler'sche Summenformel}
Sei \(n \in \mathbb{N}\) und \(f : [0, n] \to \mathbb{R}\) sei n-mal stetig differenzierbar, \(m \geq 2\). Dann gilt:

\[\sum_{k=0}^n f(k) = \int_0^n f(x) dx + \frac{1}{2}(f(0)+f(n)) + \sum_{k=2}^m \left. \frac{(-1)^k B_k}{k!} f^{(k-1)} (x) \right|_{x=0}^n + R_m\]

mit \(R_m := \frac{(-1)^{m+1}}{m!} \int_0^n f^{(m)}(x) P_m(x) dx\)
\end{theorem}
\begin{proof}
Wegen Satz 1 reicht es zu zeigen \(\int_0^n f'(x) P_1(x) dx = \sum_{k=2}^m \left. \frac{(-1)^k B_k}{k!} f^{(k-1)}(x) \right|_{x=0}^n + R_m\)

\[\int_0^n f'(x) P_1(x) dx = \sum_{k=2}^m \left. \frac{(-1)^k B_k}{k!} f^{(k-1)}(x) \right|_{x=0}^n + R_m\]
\[\int_0^n f'(x) P_1(x) dx = \left. \frac{1}{2!} f'(x) P_2(x) \right|_{x=0}^n + \left. \frac{1}{3!} f''(x) P_3(x) \right|_{x=0}^n + ...\]\[ + \left. \frac{1}{(m)!} f^{(m-1)}(x) P_{m}(x) \right|_{x=0}^n + \frac{(-1)^{m+1}}{m!} \int_0^n f^{(m)}(x) P_m(x) dx \]
\[ = \sum_{k=2}^m \frac{P_{k}(x)}{k!} f^{(k-1)}(x) + \frac{(-1)^{m+1}}{m!} \int_0^n f^{(m)}(x) P_m(x) dx \]
\end{proof}

\section{Anwendungen der Euler'schen Summenformel}
\subsection{Faulhaber Formel}

Sei \(n \in \mathbb{N}\).

\begin{align*}
1^1+2^1+3^1+5^1+...+n^1 &= \frac{n(n+1)}{2}=\frac{1}{2}n^2+\frac{1}{2}n \\
1^2+2^2+3^2+5^2+...+n^2 &= \frac{1}{3}n^3 + \frac{1}{2}n^2+\frac{1}{6}n \\
1^3+2^3+3^3+5^3+...+n^3 &= \frac{1}{4}n^4 + \frac{1}{2}n^3 + \frac{1}{4}n^2+0n \\
1^p+2^p+3^p+5^p+...+n^p &= ?
\end{align*}

\begin{theorem}
Für \(n, p \in \mathbb{N}\) gilt:

\[\sum_{k=1}^n k^p = \frac{1}{p+1} \sum_{k=0}^p {p+1 \choose k} (-1)^k B_k \cdot n^{p+1-k}\]
\end{theorem}
\begin{proof}
Wir verwenden die Eulersche  Summenformel. (Satz 2)

%TODO
\begin{align*}
\sum_{k=1}^n k^p &= \sum_{k=0}^n k^p - 0^p \\
&= \int_0^n x^p dx + \frac{1}{2}(0^p+n^p) + \sum_{k=2}^m \left. \frac{(-1)^k B_k}{k!} f^{(k-1)}(x) \right|_{x=0}^n + R_m \\
&= \left. \frac{x^{p+1}}{p+1} \right|_{x=0}^n + \frac{1}{2}n^p + \sum_{k=2}^m \left. \frac{(-1)^k B_k}{k!} f^{(k-1)}(x) \right|_{x=0}^n + \frac{(-1)^{m+1}}{m!} \int_0^n f^{(m)}(x) P_m(x) dx \\
&= \left. \frac{x^{m+1}}{m+1} \right|_{x=0}^n + \frac{1}{2}n^m + \sum_{k=2}^m \left. \frac{(-1)^k B_k}{k!} f^{(k-1)}(x) \right|_{x=0}^n + \frac{(-1)^{m+1}}{m!} \int_0^n m! P_m(x) dx \\
\end{align*}
\end{proof}

\subsection{Euler-Konstante}

Für \(n\in\mathbb{N}, H_n := \sum_{k=1}^n \frac{1}{k}\)  (n-te harmonische Zahl)
Betrachte \(x_n := H_n - ln(n)\). Diese Folge konvergiert. \(\gamma  := \lim_{n\to\infty} x_n \approx 0,577\)
\comment{
\begin{theorem}
\(x_n\) konvergiert.
\end{theorem}
\begin{proof}
\begin{align*}
x_n - x_{n-1} &= H_n - ln(n) - H_{n-1} + ln(n-1) \\
&= H_n - ln(n) - H_{n-1} + ln(n-1) \\
&= H_n - H_{n-1} - (ln(n) - ln(n-1)) \\
&= \sum_{k=1}^n \frac{1}{k} - \sum_{k=1}^{n-1} \frac{1}{k} - (ln(n) - ln(n-1)) \\
&= \frac{1}{n} - \underbrace{(ln(n) - ln(n-1))}_{\to 0\text{, da \(ln'(x) \to 0\) gilt}} \\
\end{align*}
\end{proof}
}

\begin{theorem}
\(x_n\) konvergiert.
\end{theorem}
\begin{proof}
Behauptung: \(x_n\) ist monoton fallend.
\begin{proof}
\begin{align*}
&& x_n &\geq x_{n+1} \\
\iff && x_n - x_{n+1} &\geq 0 \\
\iff && \left(\sum_{k=0}^n \frac{1}{k} - ln(n)\right) - \left(\sum\right) &\geq 0 \\
\end{align*} %TODO
\end{proof}
\end{proof}

\begin{theorem}
Es gelten folgende Gleichungen:

\begin{equation}
\gamma = \frac{1}{2} - \int_1^{\infty} \frac{P_1(x)}{x^2} dx \label{eq:1}
\end{equation}

Für \(m \geq 2\):

\begin{equation}
\gamma = \frac{1}{2} + \sum_{k=2}^m \frac{B_k}{k} - \int_1^{\infty} \frac{P_m(x)}{x^{m+1}} dx \label{eq:2}
\end{equation}

\begin{equation}
\gamma = \left( \sum_{k=1}^n\frac{1}{k} \right) - ln(n) - \frac{1}{2n} + \sum_{k=2}^m \frac{B_k}{k \cdot n^k} - \int_n^{\infty} \frac{P_m(x)}{x^{m+1}} dx \label{eq:3}
\end{equation}

\end{theorem}
\begin{proof}

%\subparagraph{zu \eqref{eq:1}}
Verwende Satz 1 mit \(f(x) := \frac{1}{x + 1}, n \geq 2\)

\[\sum_{k=1}^n \frac{1}{k} = \sum_{k=0}^{n-1} f(k) \overtext{=}{(Satz 1)} \int_0^{n-1} f(x) dx + \frac{1}{2} (f(0) + f(n-1)) + \int_0^{n-1} f'(x) P_1(x) dx\]
\[\left. \underbrace{ln(x+1)}{=ln(x)} \right|_{x=0}^{n-1} + \frac{1}{2}\left(1 + \frac{1}{n}\right) + \underbrace{\int_0^{n-1} \frac{1}{(x+1)^2} P_1(x) dx}{=\int_1^n \frac{P_1(x)}{x^2} dx}\]
\[\Longrightarrow \sum_{k=1}^n \frac{1}{k} - ln(n) = \frac{1}{2} + \frac{1}{2n} - \int_11^n \frac{P_1(x)}{x^2} dx\]

Nach Grenzwertbildung \(\lim_{n \to \infty}\) folgt die Aussage (1).

%\subparagraph{zu \eqref{eq:2}}
Verwende Satz 2 mit \(f(x) := \frac{1}{x+1}, n \geq 2\)

\[f'(x) = (-1) \cdot \frac{1}{(x+1)^2}, f''(x) = (-1)^2 \cdot 2! \frac{1}{(x+1)^3}, ..., f^{(k-1)}(x) = (-1)^{k-1} \cdot (k-1)! \frac{1}{(x+1)^k}\]
Es gilt nun

\begin{align*}
\sum_{k=1}^n \frac{1}{k} &= \sum_{k=0}^{n-1} f(k) \\
&\overtext{=}{(Satz 2)} \int_0^{n-1} f(x) dx + \frac{1}{2} \left(1 + \frac{1}{n}\right) + \sum_{k=2}^m \left. \frac{(-1)^k B_k (-1)^{k-1}\cdot(k-1)}{k! (x+1)^k} \right|_{x=0}^{n-1} \\
&\; + \frac{(-1)^{m+1}}{m!} \int_0^{n-1} P_m(x) \cdot (-1)^m \cdot m! \frac{1}{(x+1)^{m+1}} dx \\
\Longrightarrow \gamma &= \lim_{n \to \infty} \left(\sum_{k=1}^n \frac{1}{k} - ln(n)\right) = \frac{1}{2} + \sum_{k=2}^m \frac{B-k}{k} - \int_1^{\infty} \frac{P_m(x)}{x^m} dx \\
&= \frac{1}{2} + \sum_{k=2}^m \frac{B_k}{k} + \left(-\int_1^n \frac{P_m(x)}{x^m} dx\right) + \left(-\int_n^{\infty} \frac{P_m(x)}{x^m} dx\right) \\
\end{align*}
\[\Longrightarrow \sum_{k=1}^n \frac{1}{k} ln(n) + \frac{1}{2} + \frac{1}{2n} + \sum_{k=2}^m \frac{(-1) \cdot B_k}{k} \cdot \left(\frac{1}{n^k} - 1\right) = -\int_1^n \frac{P_m(x)}{x^m} dx \\
\]

Um \eqref{eq:3} zu beweisen, setzen wir folgende Darstellung in (2) ein:

Aus (*) folgt

%TODO
\begin{align*}
\gamma &= \frac{1}{2} + \sum_{k=2}^m \frac{B_k}{k} + \left(-\int_1^n \frac{P_m(x)}{x^m} dx\right) + \left(-\int_n^{\infty} \frac{P_m(x)}{x^m} dx\right) \\
\gamma &= \frac{1}{2} + \sum_{k=2}^m \frac{B_k}{k} + \left(\sum_{k=1}^n \frac{1}{k} ln(n) + \frac{1}{2} + \frac{1}{2n} + \sum_{k=2}^m \frac{(-1) \cdot B_k}{k} \cdot \left(\frac{1}{n^k} - 1\right)\right) + \left(-\int_n^{\infty} \frac{P_m(x)}{x^m} dx\right) \\
\end{align*}
\end{proof}

Man kann zeigen, dass gilt: 
\[|B_m(x)| \leq \frac{4m!}{(2\pi)^m} \forall m \in \mathbb{N}, x \in [0, 1\]

Es gilt \(B_2 = \frac{1}{6}, B_3 = 0\). Einsetzen in (3) ergibt:

\[\gamma = \underbrace{\sum_{k=1}^{10} \frac{1}{k} - ln(10) - \frac{1}{2\cdot10} + \frac{1}{12\cdot10^2}}_{} - \underbrace{\int_{10}^{\infty} \frac{P_3(x)}{x^4} dx}_{=:R}\]
\[= 0,577216... + R\]

Es gilt: \(R \leq \int_{10}^{\infty} \frac{|P_3(x)}{x^4} dx \leq 10^{-1} \cdot \underbrace{\int_{10}^{\infty} \frac{1}{x^4} dx}_{=\frac{1}{3}10^{-3}} \leq \frac{1}{3}10^{-4} \leq 4\cdot10^{-5}\)

\subsection{Stirling'sche Formel}
Seien \((x_n)_{n\in\mathbb{N}}, (y_n)_{n\in\mathbb{N}}\) zwei reelle Zahlenfolgen wobei \(y_n, x_n \neq 0\) für alle \(n \in \mathbb{N}\).
Dann heißen \((x_n), (y_n)\) asymptotisch gleich für \(h \to \infty\) genau dann, wenn

\[\lim_{n\to\infty} \frac{x_n}{y_n} = 1\]

\subsubsection{Beispiel}
\[H_n := \sum_{k=1}^n \frac{1}{k}, ln(n) =: y_n\]
\begin{theorem}
\[H_n ~ y_n, n \to \infty\]
\end{theorem}
\begin{proof}
Wissen: \(H_n - y_n \overtext{\longrightarrow}{n\to\infty} \gamma\)
%TODO
\begin{align*}
\Longrightarrow \exists_{C > 0, s_d |H_n - y_n| &\leq C \forall_{n\in\mathbb{N}} \\
\end{align*}
\end{proof}

\[x_n := n! = 1\cdot2\cdot3\cdot...\cdot n\]

\begin{theorem}
Stirling'sche Formel

Es gilt

\[n! ~ \sqrt{2 \pi n} \cdot \left(\frac{n}{e}\right)^n, n\to\infty\]

\[x_n := n^2 + n\]
\[y_n := n^2\]
%TODO
\end{theorem}

\begin{theorem}
Verwenden wir Satz 2 mit \(m = 3\)

\[f(x) := ln(x+1), n \geq 2\]

\begin{align*}
ln(n!) &= \sum_{k=1}^n ln(k) &= \sum_{k=0}^{n-1} f(k) \\
&\overtext{=}{(Satz 2)} \int_0^{n-1} f(x) dx + \frac{1}{2}(f(0)+f(n-1)) + \left. \frac{B_2}{2} f'(x) \right|_{x=0}^{n-1} + \frac{1}{3!} \int_0^{n-1} f^{(3)}(x) \cdot P_3(x) dx \\
&= \left. x \cdot ln(x) - x \right|_1^n + \left. \frac{1}{2}ln(n) + \frac{1}{12}\cdot\frac{1}{x+1} \right|_0^{n-1} + \frac{1}{3}\cdot\int_1^n \frac{1}{x^3} P_3(x) dx \\
&= n\ln(n) - n + 1 + \frac{1}{2} ln(n) + \frac{1}{12}\left(\frac{1}{n} - 1\right) + \frac{1}{3} \int_1^n \frac{P_3(x)}{x^3} dx \\
\end{align*}
\[\Longrightarrow ln(n!) - n\ln(n) + n - \frac{1}{2}ln(n) = \frac{11}{12} + \frac{1}{12n} + \frac{1}{3} \int_1^n \frac{P_3(x)}{x^3} dx\]

Wir sehen an der obigen Formel, dass \(s_n\) konvergiert, das heißt \(\exists s\in\mathbb{R}\), so dass

\[s = \lim_{n\to\infty} s_n\]

\begin{align*}
\Longrightarrow p &= e^s \\
&= \lim_{n\to\infty} e^{s_n} \\
&= \lim_{n\to\infty} \frac{e^{ln(n!)} \cdot e^n}{e^{n\ln(n)}\cdot e^{\frac{1}{2}ln(n)}} \\
&= \lim_{n\to\infty} \frac{n!e^n}{n^n\cdot\sqrt{n}} \\
&= \lim_{n\to\infty} \frac{n!}{\sqrt{n}\cdot\left(\frac{n}{e}\right)^n} \\
\end{align*}

\end{theorem}
\begin{proof}

\end{proof}

\end{document}